% -------------------------------------------------------------------
% TCC: Modelo de Trabalho Monográfico Acadêmico PUCRS 
%       para o curso de graduação em bacharelado em Engenharia de Software
% http://www.sascha-frank.com/latex-font-size.html
% Template rebuild por Leonardo Vizzotto
%
% baseado no template rebuild de Francisco Reinaldo 
%		(https://orcid.org/0000-0001-6161-6755)
%		(http://lattes.cnpq.br/7401534350061823)
%
% Versao de Controle: v0
% 
% -------------------------------------------------------------------

% --- LianTze / Reinaldo´s Fine Tuning
\RequirePackage{scrlfile}
\AfterClass{memoir}{\usepackage[a-3b]{pdfx}} %v.3
%Para funcionar, deixe na raiz: sRGB_IEC61966-2-1_black_scaled
%\BeforePackage{hyperref}{\usepackage[a-3b]{pdfx}} %v.3.2
% ---


\documentclass[
	% -- opções da classe memoir --
	12pt,				% tamanho da fonte
	oneside,			% para impressão em frente. Oposto a twoside (frente,costas)
	a4paper,			% tamanho do papel. 
	% % -- opções da classe abntex2 --
	chapter=TITLE,		% títulos de capítulos convertidos em letras maiúsculas
	% -- opções do pacote babel --
	english,			% idioma adicional para hifenização
	brazil				% o último idioma é o principal do documento
]{abntex2}


% \titlespacing*{hcommd}{hlefti}{hbefore-sepi}{hafter-sepi}[hright-sepi]
% \usepackage{titlesec}
% \titlespacing*{\chapter}{0pt}{-2\baselineskip}{-0.5}[-10pt]
% \titlespacing*{\section}{0pt}{1.1\baselineskip}{0.5\baselineskip}[-10pt]

    
% --- 
% PACOTES PARA AJUSTE e CORREÇÕES EM ABNTEX2
% --- 
\input{pacotes}


\usepackage[utf8]{inputenc}
\usepackage{float}
\usepackage{rotating}
% \usepackage{geometry}
\setlength{\marginparwidth}{2cm}


% \date{\today}
% --- 
% DADOS BASICOS DE AUTORIA
% --- 
\input{dados-gerais}
 
% --- 
% CAPA
% --- 
\input{0-capa-e-rosto/capa}
 
% --- 
% FOLHA DE ROSTO
% --- 
\input{0-capa-e-rosto/folha-de-rosto}

% --- 
% DADOS BASICOS DE AUTORIA EM ABNTEX, LOAD DE dados-gerais.tex
% --- 
% \tipotrabalho{\tipotrabalhoescrito}
\instituicao{\instituicaotrabalho}

\autor{\discente}
\titulo{\titulotrabalho}
\data{\the\year}
\local{\cidadededefesa}

\orientador{\proforientador}
%\coorientador{\profcoorientador}

% ---
% COMPILA O INDICE
% ---
\makeindex

% \documentclass[12pt,twoside]{report}
\usepackage{titlesec}
\usepackage[T1]{fontenc} 
\usepackage[utf8]{inputenc}

\begin{document}
% Format Chapter size and spaces
\titleformat{\chapter}[display]
{\normalfont\bfseries}{}{0pt}{\normalsize\thechapter.\ }
\titlespacing*{\chapter}{5pt}{-50pt}{20pt}

% Number on the section
\titleformat{\section}[hang]{\bfseries}{\thesection.\ }{0pt}{}

% Seleciona o idioma do documento (conforme pacotes do babel)
\selectlanguage{brazil}


% ----------------------------------------------------------
% ELEMENTOS PRÉ-TEXTUAIS
% ----------------------------------------------------------
% \pretextual

% ---
% Capa
% ---
\imprimircapa

% ---
% Folha de rosto
% ---
% \imprimirfolhaderosto

% ---
% Inserir folha de aprovação
% ---
% \input{1-pre-textuais/folha-de-aprovacao}

% ---
% Dedicatória
% ---
% \input{1-pre-textuais/dedicatoria}

% ---
% Agradecimentos
% ---
% \input{1-pre-textuais/agradecimentos}

% ---
% Epígrafe
% ---
\input{1-pre-textuais/epigrafe}


% ---
% RESUMOS
% ---

% Resumo em português
\input{1-pre-textuais/resumo}

% Resumo em inglês
%\input{1-pre-textuais/abstract}

% ---
% Lista de ilustrações
% ---
\pdfbookmark[0]{\listfigurename}{lof}
\listoffigures*
\cleardoublepage
% ---

% ---
% Lista de tabelas
% ---
% \pdfbookmark[0]{\listtablename}{lot}
% \listoftables*
% \cleardoublepage
% ---

% ---
% inserir lista de quadros
% ---
%\pdfbookmark[0]{\listofquadrosname}{loq}
%\listofquadros*
%\cleardoublepage
% ---

% ---
% Lista de Abreviaturas e Siglas
% ---
\input{1-pre-textuais/lista-de-abreviaturas} 
\cleardoublepage
% ---
% Lista de Símbolos
% ---
%\input{1-pre-textuais/lista-de-simbolos}

% ---
% Lista de Codigo-Fonte
% ---
%\pdfbookmark[0]{\listofcodigosname}{loc}
%\listofcodigos*
%\cleardoublepage
% ---

% ---
% Sumario
% ---
\input{1-pre-textuais/sumario.tex}


% ----------------------------------------------------------
% ELEMENTOS TEXTUAIS
% ----------------------------------------------------------
\textual
% ------------
%  INTRODUÇÃO
% ------------
\input{2-textuais/1-introducao}

%AGES 1
\input{2-textuais/AGES-1/1-introducao}
\input{2-textuais/AGES-1/2-desenvolvimento-projeto}
\input{2-textuais/AGES-1/3-atividades-desempenhadas}
\input{2-textuais/AGES-1/4-conclusao}

%AGES 2
\input{2-textuais/AGES-2/1-introducao}
\input{2-textuais/AGES-2/2-desenvolvimento-projeto}
\input{2-textuais/AGES-2/3-atividades-desempenhadas}
\input{2-textuais/AGES-2/4-conclusao}

%AGES 3
\input{2-textuais/AGES-3/1-introducao}
\input{2-textuais/AGES-3/2-desenvolvimento-projeto}
\input{2-textuais/AGES-3/3-atividades-desempenhadas}
\input{2-textuais/AGES-3/4-conclusao}

%AGES 4
\input{2-textuais/AGES-4/1-introducao}
\input{2-textuais/AGES-4/2-desenvolvimento-projeto}
\input{2-textuais/AGES-4/3-atividades-desempenhadas}
\input{2-textuais/AGES-4/4-conclusao}


%Conclusão geral
\input{2-textuais/2-conclusao}


% ----------------------------------------------------------
% Finaliza a parte no bookmark do PDF
% para que se inicie o bookmark na raiz
% e adiciona espaço de parte no Sumário
% ----------------------------------------------------------
\phantompart

% ---
% Conclusão
% ---




% ----------------------------------------------------------
% ELEMENTOS PÓS-TEXTUAIS
% ----------------------------------------------------------
\postextual


% ---
% Referências
% ---
\bibliography{referencias-acervo}

% ---
% Glossário
% ---
% \input{3-pos-textuais/glossario}

% ---
% Apêndices
% ---
\begin{apendicesenv}
% Imprime uma página indicando o início dos apêndices
% \partapendices
\input{3-pos-textuais/apendices}
\end{apendicesenv}

% ---
% INDICE REMISSIVO
% ---
\phantompart
\printindex


%---------------------------------------------------------------------
\end{document}
